%%=============================================================================
%% Samenvatting
%%=============================================================================

%% TODO: De "abstract" of samenvatting is een kernachtige (~ 1 blz. voor een
%% thesis) synthese van het document.
%%
%% Deze aspecten moeten zeker aan bod komen:
%% - Context: waarom is dit werk belangrijk?
%% - Nood: waarom moest dit onderzocht worden?
%% - Taak: wat heb je precies gedaan?
%% - Object: wat staat in dit document geschreven?
%% - Resultaat: wat was het resultaat?
%% - Conclusie: wat is/zijn de belangrijkste conclusie(s)?
%% - Perspectief: blijven er nog vragen open die in de toekomst nog kunnen
%%    onderzocht worden? Wat is een mogelijk vervolg voor jouw onderzoek?
%%
%% LET OP! Een samenvatting is GEEN voorwoord!

%%---------- Samenvatting -----------------------------------------------------
%%
%% De samenvatting in de hoofdtaal van het document

\chapter*{\IfLanguageName{dutch}{Samenvatting}{Abstract}}

%% Context: waarom is dit werk belangrijk
De maatschappij draait om data. Data zijn de gegevens die gepersisteerd zijn in een databank. Informatie is een afgeleide van deze data. Data en informatie zijn geld waard. Daarom is het van belang dat er voorzichtig met data wordt omgegaan zodat er geen gegevens verloren gaan bij mutaties in het systeem. In \glspl{RDBMS} gebruikt worden, gaat er data verloren wanneer er niet genoeg nagedacht wordt. Bij elke wijziging of verwijdering van een rij, gaat de voorgaande informatie die ter beschikking was verloren. Het probleem hierbij is dat deze databank gebruikt wordt als \gls{ssot}\footnote{\glsdesc{ssot}}}. Dit kan echter opgelost worden door \glspl{log}\footnote{\glsdesc{log}} te gebruiken, maar vermits deze niet de \gls{ssot} zijn, en er geen huidige staat van berekend wordt, is de kans groter dat deze fouten bevat of onbruikbaar is om de oude gegevens te achterhalen. Het kan ook zijn dat er gegevens niet gelogd werden door een vergeetachtigheid van een developer. Door gebruik te maken van EventSourcing gaat er geen data verloren. Dit komt omdat EventSourcing gebruik maakt van een EventStore (zie Hoofdstuk~\ref{sec:event-store}) als \gls{ssot} waar de huidige staat van berekend werd. Elke actie die gebeurt wordt eerst gepersisteerd in de EventStore, en daarna worden de projecties (zie Hoofdstuk~\ref{sec:projections}) pas uitgevoerd.

%% - Nood: waarom moest dit onderzocht worden?
Het is van belang om informatie bij te houden, dit kan altijd interessant zijn in de toekomst wanneer er specifieke businessvragen komen waar er momenteel nog geen antwoord op is.

%% - Taak: wat heb je precies gedaan?
Eerst wordt er een literatuurstudie gedaan van EventSourcing en zijn voor- en nadelen.
%% - Object: wat staat in dit document geschreven?
Daarna wordt er gekeken naar alles wat er nodig is om aan een EventSourcing applicatie te beginnen. Dit gaat van \gls{CQS}, \gls{CQRS} tot de effectieve onderdelen van EventSourcing. Er wordt ook uitleg gegeven over Skedify (Hoofdstuk~\ref{ch:skedify}) zodat de concrete businesscase duidelijk afgebakend wordt.

%% - Resultaat: wat was het resultaat?
%% - Conclusie: wat is/zijn de belangrijkste conclusie(s)?
Uit deze bachelorproef is gebleken dat EventSourcing geen meerwaarde biedt voor Skedify\footnote{Niet in de nabije toekomst, later kan dit herzien worden}.

%% - Perspectief: blijven er nog vragen open die in de toekomst nog kunnen
%%    onderzocht worden? Wat is een mogelijk vervolg voor jouw onderzoek?
Er zijn een aantal pistes die nog interessant kunnen zijn naar de toekomst toe. Dit onderzoek zou binnen 2 jaar nog eens opnieuw uitgevoerd moeten worden wanneer Skedify zijn product matuurder is geworden. Binnen 2 jaar zullen de noden van de klanten bijvoorbeeld al veel duidelijker zijn waardoor het product sterker in zijn schoenen staat. Daarnaast kan er ook onderzocht worden hoe men de overgang kan maken van het huidige systeem, naar een systeem met EventSourcing.