%%=============================================================================
%% Conclusie
%%=============================================================================

\chapter{Conclusie}
\label{ch:conclusie}

%% TODO: Trek een duidelijke conclusie, in de vorm van een antwoord op de
%% onderzoeksvra(a)g(en). Wat was jouw bijdrage aan het onderzoeksdomein en
%% hoe biedt dit meerwaarde aan het vakgebied/doelgroep? Reflecteer kritisch
%% over het resultaat. Had je deze uitkomst verwacht? Zijn er zaken die nog
%% niet duidelijk zijn? Heeft het ondezoek geleid tot nieuwe vragen die
%% uitnodigen tot verder onderzoek?

Uit dit onderzoek is gebleken dat EventSourcing zeer interessant is vanuit een business standpunt. Geen data en informatie kwijtgeraken doorheen de jaren klinkt als muziek in de oren. Het opzetten van EventSourcing binnen Skedify, dat momenteel nog zeer jong is, is veel moeilijker omdat de developers opgeleid moeten worden en de noden van de klant nog niet helemaal duidelijk zijn.

Het verschil tussen EventSourcing en systemen die gebruik maken van een \gls{RDBMS} is nu ook duidelijk. Bij EventSourcing wordt er een lijst van events bijgehouden in een append only database, de EventStore. Deze events zijn ook de \gls{ssot}. De huidige staat van de applicatie wordt berekend van deze events en in een database gevoed via projections. Bij een \gls{RDBMS} is deze databank zelf de \gls{ssot}, er worden geen events in een EventStore bijgehouden. Er worden bij sommige systemen, zoals bij Skedify, wel gebruik gemaakt van een audit log.

Het bepalen van de delen die EventSourced moeten worden kan eenvoudig door te kijken naar de belangrijke core-business delen. Bij Skedify draait dit rond de afspraken, vragen en antwoorden. Meer nog, bij Skedify zouden ze, indien ze EventSourcing zouden gebruiken in de toekomst, enkel de delen rond afspraken gaan EventSourcen omdat hier de meeste vragen van hun klanten over komen.

EventSourcing biedt momenteel geen meerwaarde voor Skedify\footnote{Het is wel interessant om dit onderzoek nog eens uit te voeren over enkele jaren, wanneer het product matuurder geworden is}.

De verwachtingen waren net iets anders, er werd verwacht dat EventSourcing wel een meerwaarde zou bieden naar Skedify toe. Uit gesprekken bij Skedify, en uit het onderzoek, is nu eenmaal gebleken dat dit niet het geval is ten opzichte van de kosten die het met zich meebrengt.

Naar de toekomst toe kan het wel interessant zijn dit nog eens opnieuw te onderzoeken. Het kan ook interessant zijn om met dit onderzoek verder te gaan en te gaan kijken hoe er effectief van de huidige manier van werken, naar EventSourcing geëvolueerd kan worden.