%%=============================================================================
%% Skedify
%%=============================================================================

\chapter{Skedify}
\label{ch:skedify}

Skedify is een Gentse tech start-up dat online afspraakbeheersoftware ontwikkelt voor onder andere de banksector, ziekenfondsen, verzekeringen, immobiliën en HR-bedrijven. Deze bachelorproef gaat op zoek naar een antwoord op de vraag of EventSourcing een meerwaarde kan bieden voor Skedify. Skedify zorgt er voor dat een persoon in eender welke sector een afspraak kan maken op een korte tijd voor eender welk probleem, en dit met de juiste contactpersoon van het bedrijf. Dit moet in een zo kort mogelijke tijd verlopen. Skedify is een business-to-business tool, die onder andere dynamisch formulieren kan gaan opbouwen om te gebruiken in de plugin die op de website van de klant terecht komt. Alle informatie binnen het systeem wordt in een relationele databank opgeslagen.

Een van de mogelijkheden die Skedify biedt is het wijzigen van een afspraak. Het kan interessant zijn om te weten hoeveel keer dit gebeurt en of er specifieke maatregelen kunnen genomen worden.

Rapportering is tot op de dag van vandaag nog niet beschikbaar omdat er geen historiek wordt bijgehouden.

Er moet nogmaals benadrukt worden dat Skedify een start-up is. Momenteel zijn er een 5-tal vaste developers in dienst. Ze werken ook volgens de \gls{agile}\footnote{\glsdesc{agile}} principes waardoor ze zo snel mogelijk proberen in te spelen op de nieuwe eisen van hun klanten. Vermits het zo een jong bedrijf is, blijkt er uit een gesprek met de product owner van Skedify, Christophe Thelen, dat nog niet alle noden van de klanten duidelijk gedefinieerd zijn. Dit heeft als gevolg dat er regelmatig wijzigingen zijn en dit kan ook invloed hebben op het kiezen van de juiste aggregates (zie Hoofdstuk~\ref{sec:aggregates}) en de juiste Domain Events (zie Hoofdstuk~\ref{sec:domain-events}).