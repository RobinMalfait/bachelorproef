%%=============================================================================
%% Nadelen
%%=============================================================================

\chapter{Nadelen}
\label{ch:nadelen}

Heel wat nadelen worden besproken in hetvolgende hoofdstuk, hoofdstuk~\ref{ch:kosten} kosten, maar met dit hoofdstuk zal er nog net iets dieper naar de nadelen gekeken worden.

Eerst en vooral is EventSourcing een nieuwe manier van denken waardoor alle developers die hiermee moeten werken, opgeleid moeten worden. Vermits Skedify een jong bedrijf is, dat momenteel full-stack developers\footnote{Developers die zowel aan de front-end als aan de back-end van een applicatie werken} aan neemt, is het moeilijk om developers te vinden die al helemaal thuis zijn in EventSourcing.

Ten tweede is het een moeilijke setup om EventSourcing werkende te krijgen binnen de applicatie. Er moet een EventStore opgezet worden, er moeten aggregates bepaald worden, er moeten Domain Events bepaald worden en er moeten projecties gemaakt worden. Daarbij komt ook nog kijken dat ze al de Domain Events moeten onderhouden doorheen de jaren.

Tot slot is het moeilijk om te bepalen welke delen er EventSourced moeten worden bij Skedify. Door een gesprek met de product owner van Skedify, Christophe Thelen, kwam hier een duidelijker beeld over. Meestal worden de core-business zaken EventSourced \autocite{Young2010WhyEventSourcing}. Maar bij Skedify zouden ze eerder naar de interessante delen van deze core-business zaken gaan, zoals bijvoorbeeld alle zaken rond de appointments zelf. Het beheren van vragen, antwoorden, customers en employees zijn ook hun core-business maar zijn minder interessant om te gaan EventSourcen.