%%=============================================================================
%% Voordelen
%%=============================================================================

\chapter{Voordelen}
\label{ch:voordelen}

Er zijn meerdere voordelen verbonden aan EventSourcing. Een voordeel is dat er van technologie gewisseld kan worden voor de projecties (zie Hoofdstuk~\ref{sec:projections}). Stel dat er gebruik gemaakt werd van \gls{mysql}\footnote{\glsdesc{mysql}} om de huidige staat in op te slaan, dan kan, wanneer er gemerkt wordt dat er nood is aan een \gls{graph}\footnote{\glsdesc{graph}} in plaats van een \glsdesc{RDBMS}, er een projectie bijgemaakt worden die simultaan met de \gls{mysql} database loopt. Zodra de \gls{graph} volledig gevoed is met de events uit de EventStore, dan kan de \gls{mysql} databank weggegooid worden. Doordat een projectie opgebouwd is in code, en in een versiebeheersysteem zit, moet er niet nagedacht worden over het nemen van een eventuele back-up van deze \gls{mysql} databank omdat ze opnieuw opgebouwd kan worden door middel van voorgaande events. Indien deze databank later opnieuw ter beschikking moet staan, dan kan deze projectie terug ingeladen worden en volledig opnieuw opgebouwd worden.
Het is mogelijk om een \gls{graph} in te voeren in de applicatie die bijvoorbeeld al meer dan 3 jaar actief is. Zodra de events volledig gevoed zijn door middel van een projectie in de \gls{graph}, dan staat deze op het punt alsof ze al van in het begin van de applicatie geprogrammeerd was. Dit is het grote voordeel van al de events bij te houden.
Een ander voordeel van EventSourcing is dat enkel de EventStore gebackupped moet worden. Indien elke server corrupt of kapot gaat en alle \gls{mysql} instanties stuk zijn, dan kunnen deze opnieuw opgebouwd worden. Dit is mogelijk omdat de EventStore de single source of truth is.