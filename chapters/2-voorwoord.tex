%%=============================================================================
%% Voorwoord
%%=============================================================================

\chapter*{Voorwoord}
\label{ch:voorwoord}

%% TODO:
%% Het voorwoord is het enige deel van de bachelorproef waar je vanuit je
%% eigen standpunt (``ik-vorm'') mag schrijven. Je kan hier bv. motiveren
%% waarom jij het onderwerp wil bespreken.
%% Vergeet ook niet te bedanken wie je geholpen/gesteund/... heeft

Ik ben al een 5 jaar lang gepassioneerd door data en informatie. Door gebruik te maken van EventSourcing gaat er geen informatie verloren, en is er een mogelijkheid om te tijdreizen naar het verleden tot op het heden. Het idee van deze bachelorproef kwam uit gesprekken met een aantal mensen via sociale media Twitter. Deze gesprekken waren met een aantal mensen die nu bekend zijn in de DDD (Domain Driven Design) en EventSourcing wereld. Een van deze mensen in Shawn McCool, die het platform https://eventsourcery.com heeft gemaakt, wat een introductie is tot domain modeling, CQRS (Command Query Responsibility Segregation, zie Hoofdstuk~\ref{ch:CQRS}) en EventSourcing (zie Hoofdstuk~\ref{ch:eventsourcing}).

Dankzij Twitter heb ik ook mijn co-promotor, Mathias Verraes leren kennen. Ik heb hem ook ontmoed op een conferentie in Nederland, namelijk Laracon EU 2014. Ik zou hem heel graag willen bedanken voor het realiseren van deze bachelorproef. Ik kon altijd met al mijn vragen raad bij hem, en hij zorgde bezorgde mij ook de nodige boeken, blog posts en andere resources omtrent EventSourcing. Ik kreeg ook de kans om naar meetups (mini conferenties waar 2 à 3-tal mensen een presentatie gegeven omtrent een bepaald onderwerp en waar je aan netwerking kan doen) te gaan, maar dit was niet altijd even gemakkelijk omdat ze niet altijd bij de deur plaatsvonden.

Ik zou ook graag de product owner van Skedify bedanken, Christophe Thelen om mij interessante gegevens over Skedify te bezorgen. Alsook resources te bezorgen in verband meet EventSourcing omdat hij hier ook al in de praktijk mee gewerkt heeft.

Daarnaast zou ik ook graag mijn promotor, Lieven Smits willen bedanken voor het benadrukken van mijn spellingsfouten om deze bachelorproef tot een goed einde te brengen.