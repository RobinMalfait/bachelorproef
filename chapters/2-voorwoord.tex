%%=============================================================================
%% Voorwoord
%%=============================================================================

\chapter*{Voorwoord}
\label{ch:voorwoord}

%% TODO:
%% Het voorwoord is het enige deel van de bachelorproef waar je vanuit je
%% eigen standpunt (``ik-vorm'') mag schrijven. Je kan hier bv. motiveren
%% waarom jij het onderwerp wil bespreken.
%% Vergeet ook niet te bedanken wie je geholpen/gesteund/... heeft

Ik ben al een 5 jaar lang gepassioneerd door data en informatie. Door gebruik te maken van EventSourcing gaat er geen informatie verloren, en is er een mogelijkheid om te tijdreizen. Het is mogelijk om een rapport te schrijven in het heden, terug te gaan naar het begin der tijden van de applicatie en daar het rapport te genereren alsof het rapport toen al bestond. Het idee van deze bachelorproef kwam voort uit gesprekken met een aantal mensen via de sociale media Twitter. Deze mensen zijn nu bekend in de \gls{DDD} en EventSourcing wereld. Een van deze mensen is Shawn McCool, die het platform https://eventsourcery.com heeft gemaakt, wat een introductie is tot domain modeling, \gls{CQRS} (zie Hoofdstuk~\ref{ch:CQRS}) en EventSourcing (zie Hoofdstuk~\ref{ch:eventsourcing}).

Dankzij Twitter heb ik ook mijn co-promotor, Mathias Verraes, leren kennen. Ik heb hem ook ontmoet op een conferentie in Nederland, namelijk Laracon EU 2014. Ik zou hem heel graag willen bedanken voor het helpen realiseren van deze bachelorproef. Ik kon altijd met al mijn vragen terecht bij hem, en hij bezorgde mij ook de nodige boeken, blog posts en andere resources omtrent EventSourcing. Ik kreeg ook de kans om naar \gls{meetup}\footnote{\glsdesc{meetup}} te gaan, maar dit was niet altijd even gemakkelijk omdat ze niet altijd bij de deur plaatsvonden.

Ik zou ook graag de product owner van Skedify, Christophe Thelen, bedanken om mij interessante gegevens over Skedify te bezorgen, alsook resources in verband met EventSourcing omdat hij hier ook al in de praktijk mee gewerkt heeft.

Daarnaast zou ik ook graag mijn promotor, Lieven Smits, willen bedanken voor het benadrukken van mijn spellingsfouten, mij te helpen bij het herstructureren van bepaalde delen en in het algemeen om deze bachelorproef tot een goed einde te brengen.

Tot slot zou ik ook graag mijn vriendin willen bedanken voor het extra nalezen van mijn bachelorproef en mij te steunen in deze drukke tijden.