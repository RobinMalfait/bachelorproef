%%=============================================================================
%% Voorwoord
%%=============================================================================

\chapter*{Voorwoord}
\label{ch:voorwoord}

%% TODO:
%% Het voorwoord is het enige deel van de bachelorproef waar je vanuit je
%% eigen standpunt (``ik-vorm'') mag schrijven. Je kan hier bv. motiveren
%% waarom jij het onderwerp wil bespreken.
%% Vergeet ook niet te bedanken wie je geholpen/gesteund/... heeft

Ik ben al een lange tijd gepassioneerd door data en informatie. Door gebruik te maken van EventSourcing gaat er geen informatie verloren, en is er een mogelijkheid om te tijdreizen naar het verleden tot op het heden. Het idee van deze bachelorproef kwam uit gesprekken met een aantal mensen via sociale media Twitter.

Dankzij Twitter heb ik ook mijn co-promotor, Mathias Verraes leren kennen en ik zou hem heel graag willen bedanken voor het realiseren van deze bachelorproef. Ik kon altijd met al mijn vragen raad bij hem, en hij zorgde er ook voor dat ik naar Meetups kon gaan om meer informatie te verschaffen voor mijn bachelorproef.

Daarnaast zou ik ook graag mijn promotor, Lieven Smits willen bedanken voor het benadrukken van mijn spellingsfouten en deze bachelorproef tot een goed einde te brengen.