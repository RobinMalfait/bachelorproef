%%=============================================================================
%% EventSourcing
%%=============================================================================

\chapter{EventSourcing}
\label{ch:eventsourcing}

EventSourcing is geen nieuwe uitvinding, maar het principe wordt gebruikt in andere sectoren zoals de bankindustrie, wetgeving, en patiënten fiches~\autocite{mann2003standards} die opgeslagen zijn bij dokters. Het is met de ideeën uit deze domeinen dat EventSourcing binnen de informatica ontstaan is (M. Verraes, persoonlijke mededeling, 25 mei 2017). De Belgische wetgeving maakt gebruik van het principe van EventSourcing in die zin dat, bij het toevoegen of ongedaan maken van wetten, er een addendum wordt toegevoegd aan de basiswetgeving (\cite{belgischewetgeving}). Dit is belangrijk zodat wanneer er een misdrijf op tijdstip $x$ wordt gepleegd en de wet op tijdstip $y$ pas verandert, er nog steeds geweten is hoe de wet er uitzag op tijdstip $x$ en er volgens de wet van tijdstip $x$ kan gestraft worden.

EventSourcing betekent dat events de bron zijn van de applicatie. Het is ook zo dat de events de \gls{ssot} zijn van de applicatie om andere \gls{state} uit af te leiden. Een log bestaat ook uit een geschiedenis van events, maar deze wordt niet gebruikt als \gls{ssot} om de huidige \gls{state} af te leiden.

EventSourcing wordt door Microsoft aanzien als een \gls{pattern}\footnote{\glsdesc{pattern}}. Microsoft heeft onderzoek gedaan naar wanneer men dit \gls{pattern} het best gebruikt en wanneer niet \autocite{Microsoft2017ES}.

Zoals eerder vermeld, is het principe van EventSourcing niet nieuw voor sectoren zoals de bankindustrie, daarom zijn er veel voorbeelden te vinden over bank transacties en e-commerce omtrent EventSourcing. Deze worden ook als voorbeeld gebruikt bij \textcite{Microsoft2017ES}. In deze bachelorproef wordt er naar Skedify gekeken, wat niet rechtstreeks te maken heeft met de standaard voorbeelden die te vinden zijn.

Als er naar de bankindustrie wordt gekeken dan is het bedrag op iemand zijn rekening niet een getal dat enkel en alleen opgeslagen is in een databank. Er is een lijst van transacties die tot dit getal komen. Het getal op een zichtrekening zit ook opgeslagen in een databank, maar dit is puur als \gls{cache}\footnote{\glsdesc{cache}} mechanisme, zodat men niet elke keer opnieuw alle transacties moet afgaan om dit getal te bepalen.

Het is zo dat EventSourcing de laatste jaren pas aan het opkomen is in de informatica sector. In de volgende hoofdstukken zal er dieper ingegaan worden op bepaalde onderdelen van EventSourcing.