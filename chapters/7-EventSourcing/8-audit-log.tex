%%=============================================================================
%% Audit Log
%%=============================================================================

\section{Audit Log}
\label{sec:audit-log}

Wanneer er gebruik gemaakt wordt van EventSourcing, is er automatisch een gratis audit log beschikbaar omdat de EventStore append-only is. Deze audit log is een bewijs van alle events die tot een bepaalde waarde (projection) leiden. Wanneer de EventStore op een \gls{WORM} drive gezet wordt, dan kan er niet geknoeid worden met deze lijst van events. Het voordeel hiervan is dat er gemakkelijk bewijs kan geleverd worden naar andere partijen indien dat nodig is. Stel dat er een rechtzaak komt tussen 2 bedrijven waarbij de andere partij beweert dat er geknoeid is met de data, dan kan er zwart op wit bewezen worden dat er niet geknoeid is met de data.

Deze audit-log heeft ook nog andere voordelen, als er een fout zit in de applicatie en de applicatie moet gedebugged worden, dan kan men de audit log nemen en deze opnieuw afspelen tot de huidige staat. Zo kan er per event gekeken worden of de uitkomst van de huidige staat al dan niet correct is (zie Hoofdstuk~\ref{sec:testen}). Dit kan ook gecontroleerd worden door een tijdelijke projectie te schrijven (zie Hoofdstuk~\ref{sec:projections}) die de huidige \gls{state} opnieuw opbouwt.

Bij Skedify doen ze momenteel al aan logging\footnote{\glsdesc{log}} van informatie. Ook de nood om bepaalde businessvragen van de klant te beantwoorden, zoals beschreven in hoofdstuk~\ref{sec:projections}, is er momenteel niet en er is dus geen nood aan een audit log. Er zijn al een aantal cases geweest waarbij dit wel het geval was, maar deze zijn tot op de dag van vandaag zo minimaal dat dit niet aan de orde is.
