%%=============================================================================
%% CQRS
%%=============================================================================

\chapter{CQRS}
\label{ch:CQRS}

\gls{CQRS} is een term uitgevonden door Greg Young. Greg Young gaat ook verder in op EventSourcing. Een goede uitleg hieromtrent is te vinden in een presentatie die hij gaf op een conferentie \autocite{Young2014CQRSandES}. \gls{CQS}, kort voor Command Query Seperation, is een principe dat uitgevonden is door Bertrand Meyer \autocite{Meyer1988}. Het is een API design principe dat beschrijft hoe er met een object of een systeem gecommuniceerd moet worden. CQRS daarentegen is een architectuurstijl die gaat over hoe dat aan de binnenkant geïmplementeerd is. Als er gekeken wordt naar \gls{CQS} is dit al een eerste vorm van goede, overzichtelijke code schrijven. \gls{CQS} zorgt er voor dat \glspl{getter}\footnote{\glsdesc{getter}} en \glspl{setter}\footnote{\glsdesc{setter}} gescheiden zijn. \Glspl{getter} zijn louter bedoeld om een waarde uit de huidige \gls{state}\footnote{\glsdesc{state}} te halen en deze terug te geven. \Glspl{setter} zijn bedoeld om een wijziging door te voeren (of een algemene actie uit te voeren), \glspl{setter} geven geen waarde terug maar \gls{void}\footnote{\glsdesc{void}}, dit principe wordt ook uitgelegd op de blog van Martin Fowler \autocite{Fowler2005CQS}. Een goede heuristiek voor het toepassen van \gls{CQS} is: ``Een vraag stellen, mag het antwoord niet wijzigen'', met andere woorden, een \gls{getter} heeft nooit gevolgen op de huidige \gls{state} of op andere zaken.

Het doel van \gls{CQS} is vooral de leesbaarheid verhogen, het is een principe dat developers gebruiken en begrijpen waardoor communicatie gemakkelijker gaat.
Het probleem is dat \glspl{getter} en \glspl{setter} in één en dezelfde klasse gedefinieerd zijn. Voor eenvoudige klassen is dit geen probleem, maar wanneer er gelezen of geschreven wordt naar een databank is dit wel een probleem. Er is geen strikte scheiding tussen de leeskant en de schrijfkant.

\codefragment{source/CQS-Appointment.java}{Voorbeeld van \gls{CQS}}

In dit voorbeeld zijn \glspl{getter} en \glspl{setter} strikt gescheiden. Het opvragen van een datum of subject wijzigt niets aan de huidige \gls{state} van dat object. Wanneer deze klasse ook verbonden is aan een databank, dan zullen er zich problemen vormen, wat te zien is in het volgende voorbeeld.

\codefragment{source/CQS-Appointment-db.java}{Voorbeeld van \gls{CQS} met databank}

Er is nu geen manier om de lees- en schrijfkant los te koppelen van elkaar. De Appointment klasse zal zowel voor het persisteren van data met deze databank verbinden, als voor het uitlezen van gegevens. Er is momenteel geen manier om te bepalen dat het lezen van data via een andere database connectie moet gaan. Het lezen en schrijven is gekoppeld aan deze klasse waardoor er geen scheiding mogelijk is.

Dit is waar \gls{CQRS}, Command Query Responsibility Segregation, komt kijken. \gls{CQRS} zorgt er voor dat de lees- en schrijfkant strikt gescheiden zijn. Het zijn bijna 2 applicaties die naast elkaar staan. Een Command wordt afgehandeld aan de schrijfkant en een Query wordt afgehandeld aan de leeskant. Zowel een Command als een Query zijn messages, dit wordt verder besproken in hoofdstuk~\ref{sec:messages}. De leeskant gaat zijn informatie halen bij de databank (of een andere vorm van opslagmechanisme), dit kan via \gls{SQL} queries, \gls{ORM} tools, enzovoort. De manier waarop dit gebeurt staat volledig los van hoe de schrijfkant communiceert met het opslagmechanisme.

De meeste applicaties, onder andere ook die van Skedify, zijn intensiever aan de leeskant dan aan de schrijfkant (Tabel \ref{cqrs-read-writes}). De lees- en schrijfkant zijn nu strikt gescheiden en er kan gebruik gemaakt worden van schalingsmechanismen om de performantie te verhogen (zie Hoofdstuk~\ref{ch:schaalbaarheid}). Beter nog, de leeskant en schrijfkant kunnen individueel geschaald worden. Er kan zelfs geopteerd worden om lees- en schrijfkant in verschillende programmeertalen te schrijven.

\begin{table}[h]
\centering
\caption[Aantal \glspl{request} vergeleken voor de lees- en schrijfkant.]{Aantal \glspl{request} vergeleken voor de lees- en schrijfkant. Er zijn ~5.548 (\textasciitilde18.02\%) keer zoveel lees \glspl{request} dan schrijf \glspl{request}\footnotemark.}
\begin{tabular}{lll} \toprule
Methode & \# \Glspl{request} & Type        \\ \midrule
GET     & 612 000     & LEZEN       \\
OPTIONS & 148 000     & LEZEN       \\
POST    & 129 000     & SCHRIJVEN   \\
HEAD    & 362 000     & LEZEN       \\
PATCH   & 13 400      & SCHRIJVEN   \\
DELETE  & 1 100       & SCHRIJVEN   \\ \bottomrule
\end{tabular}
\label{cqrs-read-writes}
\end{table}

\footnotetext{\glsdesc{request}}

\begin{figure}[h]
\caption{Visuele representatie van tabel ~\ref{cqrs-read-writes}}
\centering
\includegraphics[width=0.75\textwidth]{img/lees-en-schrijfkant}
\end{figure}

Door de zwaardere druk langs de leeskant, kan er moeilijk geoptimaliseerd worden voor alle cases waar zowel de lees- als schrijfkant voordeel uit halen. De splitsing is daarom een voordeel om langs beide kanten te kunnen optimaliseren.

Wanneer de lees- en schrijfkant niet gescheiden zouden worden, dan maken ze gebruik van de dezelfde databank. Wanneer er gelezen wordt en door een andere partij geschreven wordt, kan dit elkaar hinderen omdat beide kanten dezelfde resources gebruiken.

\gls{CQRS} speelt een grote rol bij EventSourcing, vandaar dit korte hoofdstuk.
