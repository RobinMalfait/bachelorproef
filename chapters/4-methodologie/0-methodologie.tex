%%=============================================================================
%% Methodologie
%%=============================================================================

\chapter{Methodologie}
\label{ch:methodologie}

%% TODO: Hoe ben je te werk gegaan? Verdeel je onderzoek in grote fasen, en
%% licht in elke fase toe welke stappen je gevolgd hebt. Verantwoord waarom je
%% op deze manier te werk gegaan bent. Je moet kunnen aantonen dat je de best
%% mogelijke manier toegepast hebt om een antwoord te vinden op de
%% onderzoeksvraag.


Voor er een proof of concept gemaakt kan worden in verband met EventSourcing, moet men eerst kijken naar wat EventSourcing is en wat het inhoudt. Naast EventSourcing zijn er ook concepten zoals \gls{CQRS} die even aan bod moeten komen omdat dit een interessante rol speelt bij EventSourcing. \gls{CQRS} is een splitsing van de lees- en schrijfkant in een systeem. Binnen EventSourcing heb je ook Commands (schrijfkant) en Queries (leeskant), en daarom is dit een goed hoofdstuk om binnen deze bachelorproef op te nemen. Daarnaast draait het in dit onderzoek rond Skedify (Hoofdstuk~\ref{ch:skedify}), het bedrijf die als concrete case van toepassing is voor deze bachelorproef. In het deel rond Skedify, kan er meer te weten gekomen worden over wat ze doen. Daarna zal er gekeken worden naar de huidige manier van werken bij het bedrijf omtrent hunt data. Tot slot gaat alles worden samengevoegd: Skedify en de huidige manier van werken, en Skedify en EventSourcing. Hierna worden de voor- en nadelen van beide mogelijkheden afgegaan om te zien of EventSourcing nu effectief een meerwaarde biedt voor Skedify. Het onderzoek is in grote delen opgesplitst: Skedify, EventSourcing, huidige manier van werken en de conclusies.

In het deel over Skedify zal er onderzocht worden wat ze doen, en hoe ze dit doen. Er zal ook een gesprek komen met de business kant van Skedify om vragen te kunnen beantwoorden en om duidelijke conclusies te kunnen trekken.

In het deel over EventSourcing zal er dieper onderzocht worden hoe EventSourcing juist werkt en hoe het gebruikt kan worden.