%%=============================================================================
%% Domain Events
%%=============================================================================

\section{Domain Events}
\label{sec:domain-events}

Domain events zijn een heel belangrijk onderdeel van EventSourcing. Domain events zijn de effectieve events die zullen opgeslagen worden in een append-only database. Een event is iets dat gebeurt is en nooit meer kan veranderen. Er zijn een paar heel belangrijke eigenschappen aan deze events.

\begin{itemize}
  \item{Ze bevatten een unieke id, die op voorhand vastgelegd is. Dit kan een GUID (Globally Unique Identifier) zijn.}
  \item{Ze bevatten enkel de data die gewijzigd is ten opzichte van de vorige versie.}
  \item{Alle data die ze bevatten, is correct en kan niet meer aangepast worden.}
\end{itemize}

Domain events worden opgeslagen in een append-only database, maar wat als er een fout gemaakt is gemaakt?
Indien een fout is opgetreden, moet er een nieuw domain event gemaakt worden, die het vorige event corrigeerd. Op deze manier blijft al je data correct, en verlies je geen belangrijke informatie. Er is ook niet geprutst met de historiek van deze events, wat heel belangrijk is.

Domain events hebben ook een naam, deze naam is heel specifiek in wat er gebeurt is. Het is ook belangrijk dat deze naam in de verleden tijd is opgesteld. Een event is tenslotte gebeurd. Als er in context van Skedify gesproken wordt, dan is AppointmentWasRescheduled een goede naam voor een domain event. Het is ook belangrijk dat er geen CRUD (Create Read Update Delete) events gemaakt worden zoals OfficeWasCreated, want we hebben niet effectief een office gemaakt, we hebben er een toegevoegd. Een betere naam zou zijn OfficeWasAdded.