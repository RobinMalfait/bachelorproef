%%=============================================================================
%% CQRS
%%=============================================================================

\chapter{CQRS}
\label{ch:CQRS}

CQRS is een term uitgevonden door Greg Young en is de opvolger van CQS. CQS betekend Command Query Seperation en is een principe die uitgevonden is door Bertrand Meyer (1988). Als er gekeken wordt naar CQS is dit al een eerste vorm van goede code schrijven. CQS zorgt er voor dat getters en setters gescheiden zijn. Getters zijn strict bedoeld om een waarde uit de huidige state te halen en deze terug te geven. Setters zijn bedoeld om een wijziging te doen (of een algemene actie uit te voeren), setters geven geen waarde terug maar void.

Het grootste probleem met CQS is dat developers acties uitvoeren in getters wat voor bugs kan zorgen, bijvoorbeeld het huidige project is een spel en de totale score wordt opgevraagd. Dan ziet de getter naam er als volgt uit getTotalScore, maar als deze getter een andere methode oproept zoals calculateTotalScore, en deze methode wijzigt de state, dan kan elke keer dat getTotalScore opgeroepen wordt het resultaat anders zijn.

Een tweede probleem is dat getters en setters in een en dezelfde klasse zijn gedefinieerd. Er is geen stricte scheiding tussen de lees kant en de schrijf kant.

Dit is waar CQRS komt kijken, Command Query Responsibility Segregation. CQRS zorgt er voor dat de lees kant en de schrijf kant strict gescheiden zijn. Het zijn bijna 2 applicaties die naast elkaar staan. Een command ligt aan de schrijfkant en een query aan de leeskant. De leeskant gaat zijn informatie halen bij de databank (of een andere vorm van opslagmechanisme), dit kan via sql queries, ORM tools, enzovoort. De manier waarop dit gebeurt staat volledig los van hoe de schrijfkant communiceert met het opslagmechanisme.

Heel veel applicaties, onder andere ook die van Skedify zijn veel intensiever aan de leeskant dan aan de schrijfkant. Vermits de lees- en schrijfkant nu strict gescheiden zijn kan er gebruik gemaakt worden van schalingsmechanismen. Beter nog, de leeskant en schrijfkant kunnen individueel geschaald worden.

CQRS speelt en grote rol bij EventSourcing, vandaar dit korte hoofdstuk.