%%=============================================================================
%% Methodologie
%%=============================================================================

\chapter{Methodologie}
\label{ch:methodologie}

%% TODO: Hoe ben je te werk gegaan? Verdeel je onderzoek in grote fasen, en
%% licht in elke fase toe welke stappen je gevolgd hebt. Verantwoord waarom je
%% op deze manier te werk gegaan bent. Je moet kunnen aantonen dat je de best
%% mogelijke manier toegepast hebt om een antwoord te vinden op de
%% onderzoeksvraag.


Omdat er met EventSourcing gewerkt zal worden om een proof of concept te maken, moet er eerst gekeken worden naar wat EventSourcing is en wat het allemaal inhoudt. Naast EventSourcing zijn er ook concepten zoals CQRS die aan bod moeten komen. Daarnaast draait het in dit onderzoek rond Skedify, het bedrijf die als concrete case gebruikt zal worden. In het deel rond Skedify, kan er meer te weten gekomen worden in verband met wat ze doen en hoe ze te werk gaan. Daarna zal er gekeken worden naar de huidige manier van werken. Tot slot gaan we alles samen voegen: Skedify en de huidige manier van werken, en Skedify en EventSourcing. Hierna kunnen we dan de voor- en nadelen van beide mogelijkheden afgaan om te zien of EventSourcing nu effectief een meerwaarde biedt voor Skedify. Het onderzoek is dus in grote delen opgesplitst: Skedify, EventSourcing, huidige manier van werken en de conclusies.

In het deel over Skedify zal er onderzocht worden wat ze allemaal doen, en hoe ze dit doen. Er zal ook een gesprek komen met de business kant van Skedify om vragen te kunnen beantwoorden en om duidelijke conclusies te kunnen trekken.

In het deel over EventSourcing gaan we iets dieper in hoe EventSourcing juist werkt en hoe het gebruikt kan worden.