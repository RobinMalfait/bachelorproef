%---------- Lijst afkortingen, termen -----------------------------------------

%-------------------------------- acroniemen
\newacronym[longplural={Relationele Database Management Systemen}]{RDBMS}{RDBMS}{Relationeel Database Management Systeem}
\newacronym{CQRS}{CQRS}{Command Query Responsibility Segregation}
\newacronym{CQS}{CQS}{Command Query Separation}
\newacronym{CRUD}{CRUD}{Create Read Update Delete}
\newacronym{DDD}{DDD}{Domain Driven Design/Development}
\newacronym{GUID}{GUID}{Globally Unique Identifier}
\newacronym{JSON}{JSON}{JavaScript Object Notation}
\newacronym{OOP}{OOP}{Object Oriented Programming}
\newacronym{ORM}{ORM}{Object Relational Mapper}
\newacronym{SQL}{SQL}{Structured Query Language}
\newacronym{WORM}{WORM}{Write Once Read Many}

%--------------------------------- woordenlijst

\newglossaryentry{mysql}
{
  name = MySQL,
  description = {MySQL is een managementsysteem voor relationele databases (\gls{RDBMS}), SQL is de taal die wordt gebruikt om een database van dit systeem op te bouwen, te bevragen en te onderhouden}
}

\newglossaryentry{redis}
{
  name = Redis,
  description = {Een in-memory, key-value, database.}
}

\newglossaryentry{state}
{
  name = state,
  description = {De state van een applicatie is de huidige staat waarin een applicatie zich bevindt.}
}
\newglossaryentry{pattern}
{
  name = {software design pattern},
  description = {Een software design pattern is een patroon die een herbruikbare oplossing aanbiedt voor veel voorkomende problemen binnen de context van software ontwikkeling.}
}

\newglossaryentry{graph}
{
  name = {Graph Database},
  description = {Een database waarbij er grafen structuren worden gebruikt om gegevens op te slaan in plaats van rijen en kolommen, ze maken gebruik van nodes en edges}
}

\newglossaryentry{request}
{
  name = request,
  description = {Een netwerk verzoek van een client naar een server}
}

\newglossaryentry{meetup}
{
  name = Meetup,
  description = {Mini conferenties waar 2 à 3-tal mensen een presentatie gegeven omtrent een bepaald onderwerp}
}

\newglossaryentry{log}
{
  name = Log,
  description = {Een log file is een verzameling of een opname van evenementen die gebeurd zijn in een systeem}
}

\newglossaryentry{ssot}
{
  name = ``single source of truth'',
  description = {De single source of truth is de bron die als enige vorm van waarheid aanzien wordt}
}

\newglossaryentry{agile}
{
  name = Agile,
  description = {Een softwareontwikkeling methode die iteratief en incrementeel te werk gaat}
}

\newglossaryentry{getter}
{
  name = Getter,
  description = {Een getter is een methode die informatie uit een object ophaalt, zonder dit object te muteren}
}

\newglossaryentry{setter}
{
  name = Setter,
  description = {Een setter is een methode die geen informatie uit een object ophaalt, maar het object muteert}
}

\newglossaryentry{void}
{
  name = Void,
  description = {Een type dat ``niets'' betekent}
}

\newglossaryentry{cache}
{
  name = Cache,
  description = {Een plaats waar gegevens tijdelijk worden in opgeslagen, om toegang tot data sneller te maken}
}

\newglossaryentry{command}
{
  name = Command,
  description = {Een message die de intentie heeft een actie te laten uitvoeren, ``doe dit'', ``doe dat''}
}

\newglossaryentry{commandhandler}
{
  name = {Command Handler},
  description = {Een object dat de taak zoals beschreven in een command, zal uitvoeren}
}

\newglossaryentry{query}
{
  name = Query,
  description = {Een message die de intentie heeft informatie op te vragen uit het systeem}
  plural = Queries
}

\newglossaryentry{payload}
{
  name = Payload,
  description = {Een geheel aan informatie bedoeld om getransporteerd te worden}
}

\newglossaryentry{leftfold}
{
  name = {Left Fold},
  description = {Een left fold is een functie die een lijst door middel van een andere functie reduceert tot 1 uitkomst}
}

%--------------------------------- Print me

\printglossary[type=\acronymtype,title={Lijst van acroniemen}]
\addcontentsline{toc}{chapter}{\textcolor{maincolor}{Lijst van acroniemen}}

\printglossary
\addcontentsline{toc}{chapter}{\textcolor{maincolor}{Verklarende woordenlijst}}
